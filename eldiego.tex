

	
% LaTeXar con :
%  pdflatex notebook.tex
% o bien,
%  latex notebook.dvi
%  dvipdfm notebook.dvi
%
%	" The PDF file may contain up to 25 pages of reference material, single-sided,
%   letter or A4 size, with text and illustrations readable by a person with
%   correctable eyesight without magnification from a distance of 1/2 meter. "
%
\documentclass[10pt,landscape,twocolumn,a4paper,notitlepage]{article}
\usepackage{hyperref}
\usepackage[spanish, activeacute]{babel}
\usepackage[utf8]{inputenc}
\usepackage{fancyhdr}
\usepackage{lastpage}
\usepackage{listings}
\usepackage{amssymb}
\usepackage[usenames,dvipsnames]{color}
\usepackage{graphicx}
\usepackage{wrapfig}



%%% Margenes
\setlength{\columnsep}{0.25in}    % default=10pt
\setlength{\columnseprule}{0.5pt}    % default=0pt (no line)
 
\addtolength{\textheight}{2.35in}
\addtolength{\topmargin}{-0.9in}     % ~ -0.5 del incremento anterior
 
\addtolength{\textwidth}{1.1in}
\addtolength{\oddsidemargin}{-0.55in} % -0.5 del incremento anterior
 
\setlength{\headsep}{0.08in}
\setlength{\parskip}{0in}
\setlength{\headheight}{15pt}
\setlength{\parindent}{0mm}
 
%%% Encabezado y pie de pagina
\pagestyle{fancy}
\fancyhead[LO]{\leftmark\ -\ \rightmark}
\fancyhead[C]{\textbf{\title}}
\fancyhead[RO]{P\'agina \thepage\ de \pageref{LastPage}}
\renewcommand{\headrulewidth}{0.4pt}
\fancyfoot{}
\definecolor{darkblue}{rgb}{0,0,0.4}
%%% Configuracion de Listings
\lstloadlanguages{C++}
\lstnewenvironment{code}
	{%\lstset{	numbers=none, frame=lines, basicstyle=\small\ttfamily, }%
	 \csname lst@SetFirstLabel\endcsname}
	{\csname lst@SaveFirstLabel\endcsname}
\lstset{% general command to set parameter(s)
	language=C++, basicstyle=\small\ttfamily, keywordstyle=\slshape,
	emph=[1]{tipo,usa}, emphstyle={[1]\sffamily\bfseries},
	morekeywords={tint,forn,forsn},
	basewidth={0.47em,0.40em},
	columns=fixed, fontadjust, resetmargins, xrightmargin=5pt, xleftmargin=15pt,
	flexiblecolumns=false, tabsize=2, breaklines,	breakatwhitespace=false, extendedchars=true,
	numbers=left, numberstyle=\tiny, stepnumber=1, numbersep=9pt,
	frame=l, framesep=3pt,
    basicstyle=\ttfamily,
    keywordstyle=\color{darkblue}\ttfamily,
    stringstyle=\color{magenta}\ttfamily,
    commentstyle=\color{RedOrange}\ttfamily,
    morecomment=[l][\color{OliveGreen}]{\#}
}
 
%%% Macros
\def\nbtitle#1{\begin{Large}\begin{center}\textbf{#1}\end{center}\end{Large}}
\def\nbsection#1{\section{#1}}
\def\nbsubsection#1{\subsection{#1}}
\def\nbcoment#1{\begin{small}\textbf{#1}\end{small}}
\newcommand{\comb}[2]{\left( \begin{array}{c} #1 \\ #2 \end{array}\right)}
\def\complexity#1{\texorpdfstring{$\mathcal{O}(#1)$}{O(#1)}}
 
\begin{document}

\def\title{El Diego 2.0}
 
%\begin{wrapfigure}{l}{0pt}
%\includegraphics[width=2cm]{fotodiego}
%\end{wrapfigure}


\begin{center}
\includegraphics[width=5cm]{fotodiego}
\end{center}
\tableofcontents
\newpage
 
%%% El texto propiamente dicho
\section{Algoritmos}
\textbf{\#include $<$algorithm$>$ \#include $<$numeric$>$ \\}
\begin{tabular}{|l|l|p{5.4cm}|} \hline
\textbf{Algo} & \textbf{Params} &  \textbf{Funcion} \\  \hline
%swap & e1, e2 &  da vuelta e1,e2 & $1$\\\hline
sort, stable\_sort & f, l &  ordena el intervalo \\  \hline
%is\_sorted & f, l &  \textit{bool} si esta ordenado \\  \hline
nth\_element & f, nth, l & \textit{void} ordena el n-esimo, y \\ && particiona el resto \\  \hline
fill, fill\_n & f, l / n, elem & \textit{void} llena [f, l) o [f, \\ && f+n) con elem \\  \hline
lower\_bound, upper\_bound & f, l, elem & \textit{it} al primer / ultimo donde se \\ && puede insertar elem para que\\ && quede ordenada \\  \hline
binary\_search & f, l, elem & \textit{bool} esta elem en [f, l) \\  \hline
copy & f, l, resul & hace resul+$i$=f+$i$ $\forall i$ \\  \hline
find, find\_if, find\_first\_of & f, l, elem & \textit{it} encuentra i $\in$[f,l) tq. i$=$elem, \\ & / pred / f2, l2 & pred(i), i$\in$[f2,l2)\\\hline
count, count\_if & f, l, elem/pred & cuenta elem, pred(i)\\\hline
search & f, l, f2, l2 & busca [f2,l2) $\in$ [f,l)\\\hline
replace, replace\_if & f, l, old & cambia old / pred(i) por new \\ & / pred, new &\\\hline
reverse & f, l & da vuelta\\\hline
partition, stable\_partition & f, l, pred & pred(i) ad, !pred(i) atras\\\hline
%min, max & e1, e2 & men / may & $1$\\\hline
min\_element, max\_element & f, l, [comp] & \textit{it} min, max de [f,l]\\\hline
lexicographical\_compare & f1,l1,f2,l2 & \textit{bool} con [f1,l1]<[f2,l2]\\\hline
next/prev\_permutation & f,l & deja en [f,l) la perm sig, ant\\\hline
set\_intersection, & f1, l1, f2, l2, res & [res, $\ldots$) la op. de conj\\
set\_difference, set\_union, & & \\
set\_symmetric\_difference, & &\\\hline
push\_heap, pop\_heap, & f, l, e / e / & mete/saca e en heap [f,l), \\
make\_heap & & hace un heap de [f,l)\\\hline
is\_heap & f,l & \textit{bool} es [f,l) un heap\\\hline
accumulate & f,l,i,[op] & \textit{T} $=$ $\sum$/oper de [f,l)\\\hline
inner\_product & f1, l1, f2, i & \textit{T} $=$ i $+$ [f1, l1) . [f2, $\ldots$ )\\\hline
partial\_sum & f, l, r, [op] & r+i = $\sum$/oper de [f,f+i] $\forall i \in$[f,l)\\\hline
%power & e, i, op & \textit{T} = $e^{n}$\\\hline
\_\_builtin\_ffs& unsigned int x & Pos. del primer 1 desde la derecha\\\hline
\_\_builtin\_clz & unsigned int x & Cant. de ceros desde la izquierda.\\\hline
\_\_builtin\_ctz & unsigned int x & Cant. de ceros desde la derecha.\\\hline
\_\_builtin\_popcount & unsigned int x & Cant. de 1’s en x.\\\hline
\_\_builtin\_parity & unsigned int x & 1 si x es par, 0 si es impar.\\\hline
\end{tabular}


\section{Estructuras}
\subsection{RMQ (static)}
Dado un arreglo y una operación asociativa idempotente, get(i, j) opera sobre el rango [i, j). Restricción: LVL $\ge$ 2*ceil(logn); Usar [ ] para llenar arreglo y luego build().
\begin{code}
struct RMQ{
	#define LVL 10
	tipo vec[LVL][1<<(LVL+1)];
	tipo &operator[](int p){return vec[0][p];}
	tipo get(int i, int j) {//intervalo [i,j)
		int p = 31-__builtin_clz(j-i);
		return min(vec[p][i],vec[p][j-(1<<p)]);
	}
	void build(int n) {//O(nlogn)
		int mp = 31-__builtin_clz(n);
		forn(p, mp) forn(x, n-(1<<p))
			vec[p+1][x] = min(vec[p][x], vec[p][x+(1<<p)]);
	}
};
\end{code}
\subsection{RMQ (dynamic)}
Dado un arreglo y una operación asociativa con neutro, get(i, j) opera sobre el rango [i, j).
\begin{code}
#define MAXN 100000
struct RMQ{
	static const int sz=65536;//2*2^ceil(log(n))
	tipo t[4*MAXN];
	tipo &operator[](int p){return t[sz+p];}
	void init(int n){//O(nlgn)
		sz = 1 << (32-__builtin_clz(n));
		fill(t, t+2*sz, 0); // 0=elemento neutro
	}
	void updall(){//O(n)
		dforn(i, sz) t[i]=max(t[2*i], t[2*i+1]);}
	tipo get(int i, int j, int n=1, int a=0, int b=sz/2){//O(lgn)
		if(j<=a || i>=b) return 0;//neutro
		if(i<=a && b<=j) return t[n];
		int c=(a+b)/2;
		return max(get(i, j, 2*n, a, c), get(i, j, 2*n+1, c, b));
	}
	void set(int p, tipo val){//O(lgn)
		for(p+=sz/2; p>0 && t[p]!=val;){
			t[p]=val;
			p/=2;
			val=max(t[p*2], t[p*2+1]);
		}
	}
}rmq;
//Usage:
cin >> n; rmq.init(n); forn(i, n) cin >> rmq[i]; rmq.updall();
\end{code}
\subsection{Fenwick Tree}
For 2D threat each column as a Fenwick tree, by adding a nested for in each operation
\begin{code}
struct Fenwick{
	static const int sz=1000001;
	tipo t[sz];
	tipo sum(int a, int b){return sum(b)-sum(a-1);}
	void adjust(int p, tipo v){//valid with p in [1, sz), O(lgn)
		for(; p<sz; p+=(p&-p)) t[p]+=v; }
		tipo sum(int p){//cumulative sum in [1, p], O(lgn)
		tipo s=0;
		for(; p; p-=(p&-p)) s+=t[p];
		return s;
	}
	//get largest value with cumulative sum less than or equal to x;
	//for smallest, pass x-1 and add 1 to result
	int getind(tipo x) {//O(lgn)
	  	int idx = 0, mask = N;
	  	while(mask && idx < N) {
	   		int t = idx + mask;
			if(x >= tree[t])
		  		idx = t, x -= tree[t];
	   		mask >>= 1;
	  	}
	  	return idx;
	}
};
\end{code}
\subsection{Union-Find}
\begin{code}
struct UnionFind{
	vector<int> f;//the array contains the parent of each node
	void init(int n){f.clear(); f.insert(f.begin(), n, -1);}
	int comp(int x){return (f[x]==-1?x:f[x]=comp(f[x]));}//O(1)
	void join(int i, int j) { if(comp(i)!=comp(j)) f[comp(i)] = comp(j); }
};
\end{code}
\subsection{Disjoint Intervals}
\begin{code}
bool operator< (const ii &a, const ii &b) {return a.fst<b.fst;}
//Stores intervals as [first, second]
//in case of a collision it joins them in a single interval
struct disjoint_intervals {
	set<ii> segs;
	void insert(ii v) {//O(lgn)
		if(v.snd-v.fst==0.) return;//OJO
		set<ii>::iterator it,at;
		at = it = segs.lower_bound(v);
		if (at!=segs.begin() && (--at)->snd >= v.fst)
			v.fst = at->fst, --it;
		for(; it!=segs.end() && it->fst <= v.snd; segs.erase(it++))
			v.snd=max(v.snd, it->snd);
		segs.insert(v);
	}
};
\end{code}
\subsection{RMQ (2D)}
\begin{code}
struct RMQ2D{
	static const int sz=1024;
	RMQ t[sz];
	RMQ &operator[](int p){return t[sz/2+p];}
	void build(int n, int m){//O(nm)
		forr(y, sz/2, sz/2+m)
			t[y].build(m);
		forr(y, sz/2+m, sz)
			forn(x, sz)
				t[y].t[x]=0;
		dforn(y, sz/2)
			forn(x, sz)
				t[y].t[x]=max(t[y*2].t[x], t[y*2+1].t[x]);
	}
	void set(int x, int y, tipo v){//O(lgm.lgn)
		y+=sz/2;
		t[y].set(x, v);
		while(y/=2)
			t[y].set(x, max(t[y*2][x], t[y*2+1][x]));
	}
	//O(lgm.lgn)
	int get(int x1, int y1, int x2, int y2, int n=1, int a=0, int b=sz/2){
		if(y2<=a || y1>=b) return 0;
		if(y1<=a && b<=y2) return t[n].get(x1, x2);
		int c=(a+b)/2;
		return max(get(x1, y1, x2, y2, 2*n, a, c),
         get(x1, y1, x2, y2, 2*n+1, c, b));
	}
};
//Example to initialize a grid of M rows and N columns:
RMQ2D rmq;
forn(i, M)
	forn(j, N)
		cin >> rmq[i][j];
rmq.build(N, M);
\end{code}
\subsection{Big Int}
\begin{code}
#define BASEXP 6
#define BASE 1000000
#define LMAX 1000
struct bint{
    int l;
    tipo n[LMAX];
    bint(tipo x){
        l=0;
        forn(i, LMAX){
            n[i]=x%BASE;
            x/=BASE;
            l+=!!x||!i;
        }
    }
    bint(){THIS = bint(0);}
    bint(string x){
l=(x.size()-1)/BASEXP+1;
        fill(n, n+LMAX, 0);
        tipo r=1;
        forn(i, sz(x)){
            n[i / BASEXP] += r * (x[x.size()-1-i]-'0');
            r*=10; if(r==BASE)r=1;
        }
    }
    void out(){
		cout << n[l-1];
		dforn(i, l-1) printf("%6.6llu", n[i]);//6=BASEXP!
	}
	void invar(){
		fill(n+l, n+LMAX, 0);
		while(l>1 && !n[l-1]) l--;
	}
};
bint operator+(const bint&a, const bint&b){
	bint c;
    c.l = max(a.l, b.l);
    tipo q = 0;
    forn(i, c.l) q += a.n[i]+b.n[i], c.n[i]=q %BASE, q/=BASE;
    if(q) c.n[c.l++] = q;
    c.invar();
    return c;
}
pair<bint, bool> lresta(const bint& a, const bint& b)   // c = a - b
{
	bint c;
    c.l = max(a.l, b.l);
    tipo q = 0;
    forn(i, c.l) q += a.n[i]-b.n[i], c.n[i]=(q+BASE) %BASE, q=(q+BASE)/BASE-1;
    c.invar();
    return mkp(c, !q);
}
bint& operator-= (bint& a, const bint& b){return a=lresta(a, b).fst;}
bint operator- (const bint&a, const bint&b){return lresta(a, b).fst;}
bool operator< (const bint&a, const bint&b){return !lresta(a, b).snd;}
bool operator<= (const bint&a, const bint&b){return lresta(b, a).snd;}
bool operator==(const bint&a, const bint&b){return a <= b && b <= a;}
bint operator*(const bint&a, tipo b){
    bint c;
    tipo q = 0;
    forn(i, a.l) q += a.n[i]*b, c.n[i] = q %BASE, q/=BASE;
    c.l = a.l;
    while(q) c.n[c.l++] = q %BASE, q/=BASE;
    c.invar();
    return c;
}
bint operator*(const bint&a, const bint&b){
    bint c;
    c.l = a.l+b.l;
    fill(c.n, c.n+b.l, 0);
    forn(i, a.l){
        tipo q = 0;
        forn(j, b.l) q += a.n[i]*b.n[j]+c.n[i+j], c.n[i+j] = q %BASE, q/=BASE;
        c.n[i+b.l] = q;
    }
    c.invar();
    return c;
}
pair<bint, tipo> ldiv(const bint& a, tipo b){// c = a / b ; rm = a % b
bint c;
	tipo rm = 0;
dforn(i, a.l){
        		rm = rm * BASE + a.n[i];
        		c.n[i] = rm / b;
        		rm %= b;
    }
    c.l = a.l;
    c.invar();
    return mkp(c, rm);
}
bint operator/(const bint&a, tipo b){return ldiv(a, b).fst;}
tipo operator%(const bint&a, tipo b){return ldiv(a, b).snd;}
pair<bint, bint> ldiv(const bint& a, const bint& b){
	bint c;
    bint rm = 0;
    dforn(i, a.l){
        if (rm.l==1 && !rm.n[0])
            rm.n[0] = a.n[i];
        else{
            dforn(j, rm.l) rm.n[j+1] = rm.n[j];
            rm.n[0] = a.n[i];
            rm.l++;
        }
        tipo q = rm.n[b.l] * BASE + rm.n[b.l-1];
        tipo u = q / (b.n[b.l-1] + 1);
        tipo v = q /  b.n[b.l-1] + 1;
        while (u < v-1){
            tipo m = (u+v)/2;
            if (b*m <= rm) u = m;
            else v = m;
        }
        c.n[i]=u;
        rm-=b*u;
    }
	c.l=a.l;
    c.invar();
    return mkp(c, rm);
}
bint operator/(const bint&a, const bint&b){return ldiv(a, b).fst;}
bint operator%(const bint&a, const bint&b){return ldiv(a, b).snd;}
\end{code}
\subsection{Modnum}
\begin{code}
struct mnum{
	static const tipo mod=12582917;
	tipo v;
	mnum(tipo v=0): v(v%mod) {}
	mnum operator+(mnum b){return v+b.v;}
	mnum operator-(mnum b){return v>=b.v? v-b.v : mod-b.v+v;}
	mnum operator*(mnum b){return v*b.v;}
	mnum operator^(int n){
		if(!n) return 1;
		return n%2? (*this)^(n/2)*(this) : (*this)^(n/2);}
};
\end{code}
\subsection{Bittrie}
\begin{code}
struct bitrie{
	static const int sz=1<<5;//5=ceil(log(n))
	int V;//valor del nodo
	vector<bitrie> ch;//childs
	bitrie():V(0){}//NEUTRO
	void set(int p, int v, int bit=sz>>1){//O(log sz)
		if(bit){
			ch.resize(2);
			ch[(p&bit)>0].set(p, v, bit>>1);
			V=max(ch[0].V, ch[1].V);
		}
		else V=v;
	}
	int get(int i, int j, int a=0, int b=sz){//O(log sz)
		if(j<=a || i>=b) return 0;//NEUTRO
		if(i<=a && b<=j) return V;
		if(!sz(ch)) return V;
		int c=(a+b)/2;
		return max(ch[0].get(i, j, a, c), ch[1].get(i, j, c, b));
	}
};
\end{code}
\section{Strings}
\subsection{Trie}
\begin{code}
struct Trie{
	map<char, Trie> m;
	void add(char s[]){
		if(s[0]) m[s[i]].add(s+1);
	}
	void dfs(){
		//Do stuff
		forall(it, m)
			it->second.dfs();
	}
};
\end{code}
\subsection{Suffix Array}
\begin{code}
#define MAX_N 1000
#define RABOUND(x) (x<n? RA[x] : 0)
//SA will hold the suffixes in order.
int SA[MAX_N], RA[MAX_N], n;
string s; //input string, n=sz(s)

void countingSort(int k){
	int f[MAX_N], tmpSA[MAX_N];
	zero(f);
	forn(i, n) f[RABOUND(i+k)]++;
	int sum=0;
	forn(i, max(255, n)){
		int t=f[i]; f[i]=sum; sum+=t;}
	forn(i, n)
		tmpSA[f[RABOUND(SA[i]+k)]++]=SA[i];
	memcpy(SA, tmpSA, sizeof(SA));
}
void constructSA(){//O(n log n)
	n=sz(s);
	forn(i, n) SA[i]=i, RA[i]=s[i];
	for(int k=1; k<n; k<<=1){
		countingSort(k), countingSort(0);
		int r, tmpRA[MAX_N];
		tmpRA[SA[0]]=r=0;
		forr(i, 1, n)
			tmpRA[SA[i]]=(RA[SA[i]]==RA[SA[i-1]] && RA[SA[i]+k]==RA[SA[i-1]+k] )? r : ++r;
		memcpy(RA, tmpRA, sizeof(RA));
		if(RA[SA[n-1]]==n-1) break;
	}
}
void print(){//for debug
	forn(i, n)
		cout << i << ' ' <<
		s.substr(SA[i], s.find( '$', SA[i])-SA[i]) << endl;}
\end{code}
\subsection{String Matching With Suffix Array}
\begin{code}
//returns (lowerbound, upperbound) of the search
ii stringMatching(string P){ //O(sz(P)lgn)
	int lo=0, hi=n-1, mid=lo;
	while(lo<hi){
		mid=(lo+hi)/2;
		int res=s.compare(SA[mid], sz(P), P);
		if(res>=0) hi=mid;
		else lo=mid+1;
	}
	if(s.compare(SA[lo], sz(P), P)!=0) return ii(-1, -1);
	ii ans; ans.fst=lo;
	lo=0, hi=n-1, mid;
	while(lo<hi){
		mid=(lo+hi)/2;
		int res=s.compare(SA[mid], sz(P), P);
		if(res>0) hi=mid;
		else lo=mid+1;
	}
	if(s.compare(SA[hi], sz(P), P)!=0) hi--;
	ans.snd=hi;
	return ans;
}
\end{code}
\subsection{LCP (Longest Common Prefix)}
\begin{code}
//Calculates the LCP between consecutives suffixes in the Suffix Array.
//LCP[i] is the length of the LCP between SA[i] and SA[i-1]
int LCP[MAX_N];
void computeLCP(){//O(n)
int phi[MAX_N], PLCP[MAX_N];
	phi[SA[0]]=-1;
	forr(i, 1, n) phi[SA[i]]=SA[i-1];
	int L=0;
	forn(i, n){
		if(phi[i]==-1) {PLCP[i]=0; continue;}
		while(s[i+L]==s[phi[i]+L]) L++;
		PLCP[i]=L;
		L=max(L-1, 0);
	}
	forn(i, n) LCP[i]=PLCP[SA[i]];
}
\end{code}
\section{Geometría}
\#define EPS 1e-9
\subsection{Punto}
\begin{code}
struct pto{
	tipo x, y;
	pto(tipo x=0, tipo y=0):x(x),y(y){}
	pto operator+(pto a){return pto(x+a.x, y+a.y);}
	pto operator-(pto a){return pto(x-a.x, y-a.y);}
	pto operator+(tipo a){return pto(x+a, y+a);}
	pto operator*(tipo a){return pto(x*a, y*a);}
	pto operator/(tipo a){return pto(x/a, y/a);}
	//dot product, producto interno:
	tipo operator*(pto a){return x*a.x+y*a.y;}
	//module of the cross product or vectorial product:
	//if a is less than 180 clockwise from b, a^b>0
	tipo operator^(pto a){return x*a.y-y*a.x;}
	//returns true if this is at the left side of line qr
	bool left(pto q, pto r){return ((q-*this)^(r-*this))>0;}
	bool operator<(const pto &a) const{return x<a.x || (abs(x-a.x)<EPS && y<a.y);}
bool operator==(pto a){return abs(x-a.x)<EPS && abs(y-a.y)<EPS;}
	double norm(){return sqrt(x*x+y*y);}
	tipo norm_sq(){return x*x+y*y;}
};
double dist(pto a, pto b){return (b-a).norm();}
typedef pto vec;

double angle(pto a, pto o, pto b){
	vec oa=a-o, ob=b-o;
	return acos((oa*ob) / sqrt(oa.norm_sq()*ob.norm_sq()));}

//rotate p by theta rads CCW w.r.t. origin (0,0)
pto rotate(pto p, double theta){
	return pto(p.x*cos(theta)-p.y*sin(theta),
     p.x*sin(theta)+p.y*cos(theta));
}
\end{code}
\subsection{Line}
\begin{code}
struct line{
	line() {}
	double a,b,c;//Ax+By=C
//pto MUST store float coordinates!
	line(double a, double b, double c):a(a),b(b),c(c){}
	line(pto p, pto q): a(q.y-p.y), b(p.x-q.x), c(a*p.x+b*p.y) {}
};
bool parallels(line l1, line l2){return abs(l1.a*l2.b-l2.a*l1.b)<EPS;}
pto inter(line l1, line l2){//intersection
	double det=l1.a*l2.b-l2.a*l1.b;
	if(abs(det)<EPS) return pto(INF, INF);//parallels
	return pto(l2.b*l1.c-l1.b*l2.c, l1.a*l2.c-l2.a*l1.c)/det;
}
\end{code}
\subsection{Segment}
\begin{code}
struct segm{
	pto s,f;
	segm(pto s, pto f):s(s), f(f) {}
	pto closest(pto p) {//use for dist to point
	   double l2 = dist_sq(s, f);
	   if(l2==0.) return s;
	   double t =((p-s)*(f-s))/l2;
	   if (t<0.) return s;//not write if is a line
	   else if(t>1.)return f;//not write if is a line
	   return s+((f-s)*t);
	}
	bool inside(pto p){
return ((s-p)^(f-p))==0 && min(s, f)<*this&&*this<max(s, f);}
};

bool insidebox(pto a, pto b, pto p) {
	return (a.x-p.x)*(p.x-b.x)>-EPS && (a.y-p.y)*(p.y-b.y)>-EPS;
}
pto inter(segm s1, segm s2){
	pto r=inter(line(s1.s, s1.f), line(s2.s, s2.f));
	if(insidebox(s1.s,s1.f,p) && insidebox(s2.s,s2.f,p))
			return r;
	return pto(INF, INF);
}
\end{code}
\subsection{Rectangle}
\begin{code}
struct rect{
	//lower-left and upper-right corners
	pto lw, up;
};
//returns if there's an intersection and stores it in r
bool inter(rect a, rect b, rect &r){
	r.lw=pto(max(a.lw.x, b.lw.x), max(a.lw.y, b.lw.y));
	r.up=pto(min(a.up.x, b.up.x), min(a.up.y, b.up.y));
//check case when only a edge is common
	return r.lw.x<r.up.x && r.lw.y<r.up.y;
}
\end{code}
\subsection{Polygon Area}
\begin{code}
double area(vector<tipo> &p){//O(sz(p))
	double area=0;
	forn(i, sz(p)) area+=p[i]^p[(i+1)%sz(p)];
	//if points are in clockwise order then area is negative
	return abs(area)/2;
}
//Area ellipse = M_PI*a*b where a and b are the semi axis lengths
//Area triangle = sqrt(s*(s-a)(s-b)(s-c)) where s=(a+b+c)/2
\end{code}
\subsection{Circle}
\begin{code}
vec perp(vec v){return vec(-v.y, v.x);}
line bisector(pto x, pto y){
	line l=line(x, y); pto m=(x+y)/2;
	return line(-l.b, l.a, -l.b*m.x+l.a*m.y);
}
struct Circle{
	pto o;
	double r;
//circle determined by three points, uses line
	Circle(pto x, pto y, pto z){
		o=inter(bisector(x, y), bisector(y, z));
		r=dist(o, x);
	}
	pair<pto, pto> ptosTang(pto p){
		pto m=(p+o)/2;
		tipo d=dist(o, m);
		tipo a=r*r/(2*d);
		tipo h=sqrt(r*r-a*a);
		pto m2=o+(m-o)*a/d;
		vec per=perp(m-o)/d;
		return mkp(m2-per*h, m2+per*h);
	}
};
//finds the center of the circle containing p1 and p2 with radius r
//as there may be two solutions swap p1, p2 to get the other
bool circle2PtsRad(pto p1, pto p2, double r, pto &c){
        double d2=(p1-p2).norm_sq(), det=r*r/d2-0.25;
        if(det<0) return false;
        c=(p1+p2)/2+perp(p2-p1)*sqrt(det);
        return true;
}
\end{code}
\subsection{Point in Poly}
\begin{code}
//checks if v is inside of P, using ray casting
//works with convex and concave.
//excludes boundaries, handle it separately using segment.inside()
bool inPolygon(pto v, vector<pto>& P) {
	bool c = false;
	forn(i, sz(P)){
		int j=(i+1)%sz(P);
		if((P[j].y>v.y) != (P[i].y > v.y) &&
	(v.x < (P[i].x - P[j].x) * (v.y-P[j].y) / (P[i].y - P[j].y) + P[j].x))
			c = !c;
	}
	return c;
}
\end{code}
\subsection{Convex Check CHECK}
\begin{code}
bool isConvex(vector<int> &p){//O(N)
	int N=sz(p);
	if(N<3) return false;
	bool isLeft=p[0].left(p[1], p[2]);
	forr(i, 1, N)
		if(p[i].left(p[(i+1)%N], p[(i+2)%N])!=isLeft)
			return false;
	return true; }
\end{code}
\subsection{Convex Hull}
\begin{code}
//stores convex hull of P in S, CCW order
void CH(vector<pto>& P, vector<pto> &S){
	S.clear();
	sort(P.begin(), P.end());
	forn(i, sz(P)){
		while(sz(S)>= 2 && S[sz(S)-1].left(S[sz(S)-2], P[i])) S.pop_back();
		S.pb(P[i]);
	}
	S.pop_back();
	int k=sz(S);
	dforn(i, sz(P)){
		while(sz(S) >= k+2 && S[sz(S)-1].left(S[sz(S)-2], P[i])) S.pop_back();
		S.pb(P[i]);
	}
	S.pop_back();
}
\end{code}
\subsection{Cut Polygon}
\begin{code}
//cuts polygon Q along the line ab
//stores the left side (swap a, b for the right one) in P
void cutPolygon(pto a, pto b, vector<pto> Q, vector<pto> &P){
	P.clear();
	forn(i, sz(Q)){
		double left1=(b-a)^(Q[i]-a), left2=(b-a)^(Q[(i+1)%sz(Q)]-a);
		if(left1>=0) P.pb(Q[i]);
		if(left1*left2<0)
			P.pb(inter(line(Q[i], Q[(i+1)%sz(Q)]), line(a, b)));
	}
}
\end{code}
\subsection{Bresenham}
\begin{code}
//plot a line approximation in a 2d map
void bresenham(pto a, pto b){
	pto d=b-a; d.x=abs(d.x), d.y=abs(d.y);
	pto s(a.x<b.x? 1: -1, a.y<b.y? 1: -1);
	int err=d.x-d.y;
	while(1){
		m[a.x][a.y]=1;//plot
		if(a==b) break;
		int e2=2*err;
		if(e2 > -d.y){
			err-=d.y, a.x+=s.x;
		if(e2 < d.x)
			err+= d.x, a.y+= s.y;
	}
}
\end{code}
\subsection{Rotate Matrix}
\begin{code}
//rotates matrix t 90 degrees clockwise
//using auxiliary matrix t2(faster)
void rotate(){
	forn(x, n) forn(y, n)
		t2[n-y-1][x]=t[x][y];
	memcpy(t, t2, sizeof(t));
}
\end{code}
\section{Math}
\subsection{GCD}
\begin{code}
tipo gcd(tipo a, tipo b){return a?gcd(b %a, a):b;}
\end{code}
\subsection{LCM}
\begin{code}
tipo lcm(tipo a, tipo b){return a*b/gcd(a, b);}
\end{code}
\subsection{Simpson}
\begin{code}
double integral(double a, double b, int n=10000) {//O(n), n=cantdiv
	double area=0, h=(b-a)/n, fa=f(a), fb;
	forn(i, n){
		fb=f(a+h*(i+1));
		area+=fa+ 4*f(a+h*(i+0.5)) +fb, fa=fb;
	}
	return area*h/6.;}
\end{code}
\subsection{Fraction}
\begin{code}
struct frac{
	tipo p,q;
	frac(tipo p=0, tipo q=1):p(p),q(q) {norm();}
	tipo mcd(tipo a, tipo b){return a?mcd(b %a, a):b;}
	void norm(){
		tipo a = mcd(p,q);
		if(a) p/=a, q/=a;
		else q=1;
		if (q<0) q=-q, p=-p;}
	frac operator+(const frac& o){
		tipo a = mcd(q,o.q);
		return frac(p*(o.q/a)+o.p*(q/a), q*(o.q/a));}
	frac operator-(const frac& o){
		tipo a = mcd(q,o.q);
		return frac(p*(o.q/a)-o.p*(q/a), q*(o.q/a));}
	frac operator*(frac o){
		tipo a = mcd(q,o.q), b = mcd(o.q,p);
		return frac((p/b)*(o.p/a), (q/a)*(o.q/b));}
	frac operator/(frac o){
		tipo a = mcd(q,o.q), b = mcd(o.p,p);
		return frac((p/b)*(o.q/a),(q/a)*(o.p/b));}
	bool operator<(frac o){return (*this-o).p<0;}
	bool operator==(frac o){return p==o.p&&q==o.q;}
};
\end{code}
\subsection{Polinomio}
\begin{code}
#define MAX_GR  20
struct poly {
	int p[MAX_GR];//guarda los coeficientes del polinomio
	poly(){zero(p);}
	int gr(){//calculates grade of the polynomial
		dforn(i,MAX_GR) if(p[i]) return i;
		return 0; }
	bool isnull() {return gr()==0 && !p[0];}
	poly operator+(poly b) {// - is analogous
		poly c=THIS;
		forn(i,MAX_GR) c.p[i]+=b.p[i];
		return c;
	}
	poly operator*(poly b) {
		poly c;
		forn(i,MAX_GR) forn(k,i+1) c.p[i]+=p[k]*b.p[i-k];
		return c;
	}
	int eval(int v) {
		int sum = 0;
		forn(i,MAX_GR) sum+=p[i]*pow(v,i);
		return sum;
	}
	//the following function generates the roots of the polynomial
//it can be easily modified to return float roots
	set<int> roots(){
		set<int> roots;
		int a0 = abs(p[0]), an = abs(p[gr()]);
		vector<int> ps,qs;
		forr(p,1,sqrt(a0)+1) if (a0%p==0) ps.pb(p),ps.pb(a0/p);
		forr(q,1,sqrt(an)+1) if (an%q==0) qs.pb(q),qs.pb(an/q);
		forall(pt,ps)
			forall(qt,qs) if ( (*pt) % (*qt)==0 ) {
				int root = abs((*pt) / (*qt));
				if (eval(root)==0) roots.insert(root);
			}
		return roots;
	}
};
//the following functions allows parsing an expression like
//34+150+4*45
//into a polynomial(el numero en funcion de la base)
#define LAST(s) (sz(s)? s[sz(s)-1] : 0)
#define POP(s) s.erase(--s.end());
poly D(string &s) {
	poly d;
	for(int i=0; isdigit(LAST(s)); i++) d.p[i]=LAST(s)-'0', POP(s);
	return d;}

poly T(string &s) {
	poly t=D(s);
	if (LAST(s)=='*'){POP(s); return T(s)*t;}
	return t;
}
//main function, call this to parse
poly E(string &s) {
	poly e=T(s);
	if (LAST(s)=='+'){POP(s); return E(s)+e;}
	return e;
}
\end{code}
\section{Grafos}
\subsection{Dijkstra}
\begin{code}
#define INF 1e9
int N;
#define MAX_V 250001
vector<ii> G[MAX_V];
//To add an edge use
#define add(a, b, w) G[a].pb(mkp(w, b))

ll dijkstra(int s, int t){//O(|E| log |V|)
	priority_queue<ii, vector<ii>, greater<ii> > Q;
	vector<ll> dist(N, INF); vector<int> dad(N, -1);
	Q.push(mkp(0, s)); dist[s] = 0;
	while(sz(Q)){
		ii p = Q.top(); Q.pop();
		if(p.snd == t) break;
		forall(it, G[p.snd])
			if(dist[p.snd]+it->first < dist[it->snd]){
				dist[it->snd] = dist[p.snd] + it->fst;
				dad[it->snd] = p.snd;
				Q.push(mkp(dist[it->snd], it->snd));
			}
	}
	return dist[t];
	if(dist[t]<INF)//path generator
		for(int i=t; i!=-1; i=dad[i])
			printf("%d%c", i, (i==s?'\n':' '));
}
\end{code}
\subsection{Bellman-Ford}
\begin{code}
vector<ii> G[MAX_N];//ady. list with pairs (weight, dst)
int dist[MAX_N];
void bford(int src){//O(VE)
	dist[src]=0;
	forn(i, N-1) forn(j, N) if(dist[j]!=INF) forall(it, G[j])
		dist[it->snd]=min(dist[it->snd], dist[j]+it->fst);
}

bool hasNegCycle(){
	forn(j, N) if(dist[j]!=INF) forall(it, G[j])
		if(dist[it->snd]>dist[j]+it->fst) return true;
	//inside if: all points reachable from it->snd will have -INF distance(do bfs)
	return false;
}
\end{code}
\subsection{Floyd-Warshall}
G[i][j] contains weight of edge (i, j) or INF
G[i][i]=0
\begin{code}
int G[MAX_N][MAX_N];
void floyd(){//O(N^3)
forn(k, N) forn(i, N) if(G[i][k]!=INF) forn(j, N) if(G[k][j]!=INF)
	G[i][j]=min(G[i][j], G[i][k]+G[k][j]);
}
bool inNegCycle(int v){
	return G[v][v]<0;}
//checks if there's a neg. cycle in path from a to b
bool hasNegCycle(int a, int b){
	forn(i, N) if(G[a][i]!=INF && G[i][i]<0 && G[i][b]!=INF)
		return true;
	return false;
}
\end{code}
\subsection{2-SAT + Tarjan SCC}
We have a vertex representing a var and other for his negation.
Every edge stored in G represents an implication. To add an equation of the form a||b, use addor(a, b)
MAX=max cant var, n=cant var
\begin{code}
#define addor(a, b) (G[neg(a)].pb(b), G[neg(b)].pb(a)) 
vector<int> G[MAX*2];
//idx[i]=index assigned in the dfs
//lw[i]=lowest index(closer from the root) reachable from i
int lw[MAX*2], idx[MAX*2], qidx;
stack<int> q;
int qcmp, cmp[MAX*2];
//verdad[cmp[i]]=valor de la variable i
bool verdad[MAX*2+1];

int neg(int x) { return x>=n? x-n : x+n;}
void tjn(int v){
	lw[v]=idx[v]=++qidx;
	q.push(v), cmp[v]=-2;
	forall(it, G[v]){
		if(!idx[*it] || cmp[*it]==-2){
			if(!idx[*it]) tjn(*it);
			lw[v]=min(lw[v], lw[*it]);
		}
	}
	if(lw[v]==idx[v]){
		qcmp++;
		int x;
		do{x=q.top(); q.pop(); cmp[x]=qcmp;}while(x!=v);
		verdad[qcmp]=(cmp[neg(v)]<0);
	}
}
//remember to CLEAR G!!!
bool satisf(){//O(n)
	memset(idx, 0, sizeof(idx)), qidx=0;
	memset(cmp, -1, sizeof(cmp)), qcmp=0;
	forn(i, n){
		if(!idx[i]) tjn(i);
		if(!idx[neg(i)]) tjn(neg(i));
	}
	forn(i, n) if(cmp[i]==cmp[neg(i)]) return false;
	return true;
}
\end{code}
\subsection{Articulation Points}
\begin{code}
int N;
vector<int> G[1000000];
//V[i]=node number(if visited), L[i]= lowest V[i] reachable from i
int qV, V[1000000], L[1000000], P[1000000];
void dfs(int v, int f){
	L[v]=V[v]=++qV;
	forall(it, G[v])
		if(!V[*it]){
			dfs(*it, v);
			L[v] = min(L[v], L[*it]);
			P[v]+= L[*it]>=V[v];
		}
		else if(*it!=f)
			L[v]=min(L[v], V[*it]);
}
int cantart(){ //O(n)
	qV=0;
	zero(V), zero(P);
	dfs(1, 0); P[1]--;
	int q=0;
	forn(i, N) if(P[i]) q++;
return q;
}
\end{code}
\subsection{LCA + Climb}
\begin{code}
LCA
#define POT2(x) (1<<(x))
//f[v][k] holds the 2^k father of v
//L[v] holds the level of v
int N, f[100001][20], L[100001];
void build(){//f[i][0] must be filled previously, O(nlgn)
	forn(k, 20-1) forn(i, N) f[i][k+1]=f[f[i][k]][k];}
int lg(int x){//=floor(log2(x))
	int i;
	for (i=0;(1<<i)<=x;i++) ;
	return i-1;
}
int climb(int a, int d){//O(lgn)
	if(!d) return a;
	dforn(i, lg(L[a])+1)
		if(POT2(i)<=d)
			a=f[a][i], d-=POT2(i);
    return a;
}
int lca(int a, int b){//O(lgn)
	if(L[a]<L[b]) swap(a, b);
	a=climb(a, L[a]-L[b]);
	if(a==b) return a;
	dforn(i, lg(L[a])+1)
		if(f[a][i]!=f[b][i])
			a=f[a][i], b=f[b][i];
	return f[a][0];
}
\end{code}
\section{Network Flow}
\subsection{Edmonds Karp’s}
\begin{code}
#define MAX_V 1000
#define INF 1e9
//special nodes
#define SRC 0
#define SNK 1
map<int, int> G[MAX_V];//limpiar esto
//To add an edge use
#define add(a, b, w) G[a][b]=w
int f, p[MAX_V];
void augment(int v, int minE){
	if(v==SRC) f=minE;
	else if(p[v]!=-1){
		augment(p[v], min(minE, G[p[v]][v]));
		G[p[v]][v]-=f, G[v][p[v]]+=f;
	}
}
ll maxflow(){//O(VE^2)
	ll Mf=0;
	do{
		f=0;
		char used[MAX_V]; queue<int> q; q.push(SRC);
		zero(used), memset(p, -1, sizeof(p));
		while(sz(q)){
			int u=q.front(); q.pop();
			if(u==SNK) break;
			forall(it, G[u])
				if(it->snd>0 && !used[it->fst])
					used[it->fst]=true, q.push(it->fst), p[it->fst]=u;
		}
		augment(SNK, INF);
		Mf+=f;
	}while(f);
	return Mf;
}
\end{code}
\subsection{Push-Relabel}
\begin{code}
#define MAX_V 1000
int N;//valid nodes are [0...N-1]
#define INF 1e9
//special nodes
#define SRC 0
#define SNK 1
map<int, int> G[MAX_V];
//To add an edge use
#define add(a, b, w) G[a][b]=w
ll excess[MAX_V];
int height[MAX_V], active[MAX_V], count[2*MAX_V+1];
queue<int> Q;
void enqueue(int v) { 
	if (!active[v] && excess[v] > 0) active[v]=true, Q.push(v); }
void push(int a, int b) {
	int amt = min(excess[a], ll(G[a][b]));
	if(height[a] <= height[b] || amt == 0) return;
	G[a][b]-=amt, G[b][a]+=amt;
	excess[b] += amt, excess[a] -= amt;
	enqueue(b);
}
void gap(int k) {
	forn(v, N){
		if (height[v] < k) continue;
		count[height[v]]--;
		height[v] = max(height[v], N+1);
		count[height[v]]++;
		enqueue(v);
	}
}
void relabel(int v) {
	count[height[v]]--;
	height[v] = 2*N;
	forall(it, G[v])
		if(it->snd)
			height[v] = min(height[v], height[it->fst] + 1);
	count[height[v]]++;
	enqueue(v);
}
ll maxflow() {//O(V^3)
	zero(height), zero(active), zero(count), zero(excess);
	count[0] = N-1;
	count[N] = 1;
	height[SRC] = N;
	active[SRC] = active[SNK] = true;
	forall(it, G[SRC]){
		excess[SRC] += it->snd;
		push(SRC, it->fst);
	}
	while(sz(Q)) {
		int v = Q.front(); Q.pop();
		active[v]=false;
	forall(it, G[v]) push(v, it->fst);
	if(excess[v] > 0) 
		count[height[v]] == 1? gap(height[v]):relabel(v);
	}
	ll mf=0;
	forall(it, G[SRC]) mf+=G[it->fst][SRC];
	return mf;
}
\end{code}
\section{Ayudamemoria}
\subsection*{Límites}
\begin{code}
#include <climits> //INT_MIN, LONG_MAX, ULLONG_MAX, etc.
\end{code}
\subsection*{Cant. decimales}
\begin{code}
#include <iomanip>
cout << setprecision(2) << fixed;
\end{code}
\subsection*{Rellenar con espacios(para justificar)}
\begin{code}
#include <iomanip>
cout << setfill(' ') << setw(3) << 2 << endl;
\end{code}
\subsection*{Leer hasta fin de línea}
\begin{code}
#include <sstream>
//hacer cin.ignore() antes de getline()
while(getline(cin, line)){
   	 istringstream is(line);
   	 while(is >> X)
   		 cout << X << " ";
   	 cout << endl;
}
\end{code}
\subsection*{Aleatorios}
\begin{code}
#define RAND(a, b) (rand()%(b-a+1)+a)
srand(time(NULL));
\end{code}
\subsection*{Muahaha}
\begin{code}
#include <climits> //INT_MIN, LONG_MAX, ULLONG_MAX, etc.
\end{code}
\subsection*{Límites}
\begin{code}
#include <signal.h>
void divzero(int p){
	while(true);}
void segm(int p){
	exit(0);}
//in main
signal(SIGFPE, divzero);
signal(SIGSEGV, segm);
\end{code}
\subsection*{Mejorar velocidad}
\begin{code}
ios::sync_with_stdio(false);
\end{code}
\subsection*{Leer del teclado}
\begin{code}
freopen("/dev/tty", "a", stdin);
\end{code}
\subsection*{File setup}
\begin{code}
//tambien se pueden usar comas: {a, x, m, l}
for i in {a..k}; do cp template.cpp $i.cpp; touch $i.in; done
\end{code}
\end{document}
